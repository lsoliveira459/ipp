\documentclass[11pt,a4paper]{article} 

\usepackage[brazilian]{babel}
\usepackage[utf8]{inputenc}
\usepackage{listings}
\usepackage{color}
\usepackage{xcolor}
\usepackage{parskip}
\usepackage{fullpage}

\title{{\large Capítulo 03}\\\textit{Greetings!}}
\author{Lucas Sousa de Oliveira (10/59491)}
\date{\today}

\lstdefinestyle{customc}{
  belowcaptionskip=1\baselineskip,
  breaklines=true,
  frame=l,
  xleftmargin=\parindent,
  language=C,
  showstringspaces=false,
  basicstyle=\footnotesize\ttfamily,
  keywordstyle=\bfseries\color{green!40!black},
  commentstyle=\itshape\color{purple!40!black},
  identifierstyle=\color{blue!50!black},
  stringstyle=\color{orange},
  numbers=left,
  stepnumber=1,
}

\begin{document}
\maketitle

\section*{Objetivo}
Introduzir o aluno ao processamento paralelo através do estudo de um programa simples chamado \textit{\lq\lq{}Greetings!\rq\rq{}}, equivalente ao famoso \textit{\lq\lq{}Hello World!\rq\rq{}}.

\section*{Resumo}

\begin{lstlisting}[style=customc]
#include <stdio.h>
#include <string.h>
#include "mpi.h"

int main(int argc, char* argv[]) {
  int         my_rank;
  int         p;
  int         source;
  int         dest;
  int         tag = 0;
  char        message[100];
  MPI_Status  status;

  MPI_Init(&argc, &argv);
  MPI_Comm_rank(MPI_COMM_WORLD, &my_rank);
  MPI_Comm_size(MPI_COMM_WORLD, &p);

  if (my_rank != 0) {
    sprintf(message, "Greetings from process %d!", my_rank);
    dest = 0;
    MPI_Send(message, strlen(message)+1, MPI_CHAR, dest, tag, MPI_COMM_WORLD);
  } else {
    for (source = 1; source < p; source++) {
        MPI_Recv(message, 100, MPI_CHAR, source, tag, MPI_COMM_WORLD, &status);
        printf("%s\n", message);
    }
  }

  MPI_Finalize();
}
\end{lstlisting}

\section*{Exercícios}
\subsection*{1. Crie um arquivo contendo o programa \textit{\lq\lq{}Greetings!\rq\rq{}}. Descubra como compilar e executar em um número diferente de processadores. Qual é a saída do programa se ele for executado com apenas um processo? Quantos processadores podem ser usados?}
\subsection*{2. Modifique o programa \textit{\lq\lq{}Greetings!\rq\rq{}} de forma que ele use elementos genéricos para \textit{source} e \textit{tag}. Existe alguma diferença na saída do programa?}
\subsection*{3. Tente modificar alguns parametros de MPI\_Send e MPI\_Recv (e.g., \textit{count}, \textit{datatype}, \textit{source}, \textit{dest}). O que acontece quando se executa o programa? Ele falha? Ele pára?}
\subsection*{4. Modifique o programa \textit{\lq\lq{}Greetings!\rq\rq{}} de forma que todos os processos enviam uma mensagem para o processo $\textbf{p-1}$.}

\section*{Trabalho de programação}
\subsection*{1. Escreva um programa em que cada processo $\textbf{i}$ envia mensagem ao processo $\textbf{(i+1)\%p}$. (Cuidado em como $\textbf{i}$ calcula de quem deve receber a mensagem). O processo $\textbf{i}$ deverá enviar uma mensagem para $\textbf{i+1}$ e depois receber de $\textbf{i-1}$, ou o contrário? Faz diferença? O que acontece quando o programa é rodado em um processador?}

\end{document}
